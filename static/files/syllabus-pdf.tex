\documentclass{article}
\usepackage{booktabs, graphicx, hyperref, fontspec}
\usepackage{sectsty}
\allsectionsfont{\sffamily}
\usepackage[margin=1in]{geometry}
\hypersetup{
  colorlinks = true,
  urlcolor = cyan,
 }
 \providecommand{\tightlist}{%
  \setlength{\itemsep}{0pt}\setlength{\parskip}{0pt}}
\newcommand*{\authorfont}{\fontfamily{phv}\selectfont}
\usepackage[]{Fira Sans}

\begin{document}

\sffamily

\centerline{\Huge Economics of the Law}

\vspace{3 mm}

\centerline{\large Dr.~Ryan Safner}
\vspace{2 mm}
\centerline{\large \href{http://lawS21.classes.ryansafner.com}{lawS21.classes.ryansafner.com}}

\vspace{5 mm}

\begin{tabular}{@{}p{3.5in}p{3.5in}}           
\textbf{Course}: ECON 315 Spring
2021  & \textbf{Email:}  \href{mailto:safner@hood.edu}{\nolinkurl{safner@hood.edu}}\\
\textbf{Room}:  ON ZOOM & \textbf{Office:}  Rosenstock 118\\
\textbf{Meets}: TuTh 11:30 AM--12:45PM & \textbf{Hours:}  TuTh
3:30--5:00PM on Zoom\\ 
\end{tabular}

\vspace{5 mm}

\hrule


\begin{quote}
``For the rational study of the law the blackletter man may be the man
of the present, but the man of the future is the man of statistics and
the master of economics\ldots As a step toward that ideal it seems to me
that every lawyer ought to seek an understanding of economics\ldots We
learn that for everything we have we give up something else, and we are
taught to set the advantage we gain against the other advantage we lose,
and to know what we are doing when we elect.''\footnote{Holmes, Oliver
  Wendell, Jr., 1897, ``The Path of the Law,'' 10 Harvard Law Review
  457:12-16} {--- Oliver Wendell Holmes, Jr.}
\end{quote}

{The law}{[}\^{}courseinfo{]}, in addition to promoting justice and
redressing \emph{past} wrongs, has a strong affect on the allocation of
scarce resources and incentives for \emph{future} decisions. The
analysis of how legal rules and judicial decisions impact \emph{economic
efficiency} has been most strongly associated with the
\href{https://en.wikipedia.org/wiki/Law_and_economics}{``law and
economics'' tradition} of economists and lawyers since the 1960s. This
tradition has grown rapidly, with many top law schools and judicial
opinions now explicitly discussing the effects on incentives and the
economic efficiency of legal rulings. Economist Ronald Coase won the
1991 Nobel Prize in Economics in part for his foundational papers in
this tradition, and several influential federal judges have explicitly
applied this approach, such as Richard Posner, Guido Calabresi, and
Frank Easterbrook.

This course is about the economics of the law and economic function of
legal systems, primarily common law in the United States. We take an
economic approach to understanding and appraising law from both positive
and normative standpoints in accordance with the ``law and economics''
tradition.

\textbf{This is \emph{not} a law course, this is an upper-level
\emph{economics} elective that applies economic models and tools to
legal subjects.} This course does \emph{not} provide a survey of all of
the areas of law. For a full survey of different areas of the law, with
a focus on legal cases and jurisprudence, see \textbf{LWPS 230 ---
Introduction to Law}. This course also does \emph{not} cover any
constitutional law, or legal interpretations of economic regulation. If
you are looking for a course on the constitutional history of economic
regulation, particularly through famous Supreme Court cases, see
\textbf{PS 307 --- American Constitutional Law.} If you are looking for
a course on the economic aspects of regulation, see \textbf{ECON 326 ---
Industrial Organization.} In general, this is not a class \emph{about}
the law or its evolution, it's a class in how to \emph{think about} the
law, using economics.

This course will examine a small number of influential court cases, but
this is primarily an \emph{economics} course about modelling
\emph{incentives} and \emph{consequences}. As such, we will focus on the
operation of economic logic of the institutions, and incentives
underpinning the laws of property, contracts, torts, and criminal law,
and how we might change them to achieve the goals we desire as a
society. Finally, we compare the Anglo-American common law legal system
to other legal systems in their methods and effectiveness of achieving
these goals. The economic approach to law applies many core
microeconomic models and concepts\footnote{Such as rational choice
  theory, behavioral economics, monopoly, perfect competition,
  oligopoly, the Coase Theorem, welfare analysis, constrained
  optimization models, price theory, equilibrium models, cost-benefit
  analysis, expected utility models, game theory, etc} as well as
empirical research to legal rules in these areas. As such, the
\textbf{required prerequisite} for this course is \textbf{ECON 206 ---
Principles of Microeconomics}, with \textbf{ECON 306 --- Microeconomic
Analysis recommended}.

\textbf{Legal Disclaimer}: Although I am married to one, I am not an
attorney. No aspect of this course should be construed as legal advice
on any subject matter. You should not act or refrain from acting on the
basis of any content included in this course without seeking legal
counsel.

\hypertarget{hybrid-course-format}{%
\section*{Hybrid Course Format}\label{hybrid-course-format}}
\addcontentsline{toc}{section}{Hybrid Course Format}

This course is taught in a \textbf{hybrid} format, providing a mixture
of regular synchronous activity where we all can interact in real time,
with asynchronous material, which can be done remotely at your own pace.

\textbf{I will be holding all synchronous class sessions remotely (for
reasons I will make clear to you by the first day) on Zoom.} You can
attend these sessions on your computer or device from your dormitory or
home, and a classroom is available for you to use (socially-distanced,
and in masks), but I will not be in the classroom.

During the synchronous, scheduled times for the course (Tuesday/Thursday
11:30 A.M.-12:45 P.M.), I will lecture on the material, hold in-class
discussions, and answer questions in real time \emph{on Zoom.}
Attendance to the live portion is \emph{strongly encouraged}, but not
required.

\textbf{Lecture slides, videos, and other synchronous materials will be
posted online by the time live sessions occur.} There will be occasional
exceptions.

\textbf{Assignments will always be submitted \emph{online}} and due at
regular times (typically 11:59 PM Sunday) so that students unable to
join in the live sessions can complete them asynchronously.

Students are strongly encouraged to join the course
\href{https://hoodcollegeeconomics.slack.com}{Slack channel} to maintain
an active channel of communication, ask questions, and to build our
course community together. Official course-related announcements will
always come via Blackboard announcement and automatically sent to your
Hood email accounts.

\hypertarget{learning-in-a-time-of-coronavirus}{%
\subsection*{Learning in a Time of
Coronavirus}\label{learning-in-a-time-of-coronavirus}}
\addcontentsline{toc}{subsection}{Learning in a Time of Coronavirus}

Everything is awful right now. None of us signed up for this. None of us
are really okay, \textbf{we're all just pretending for everyone else.}

Many of you may be dealing with hardships at home and at work, and are
generally juggling many more problems than usual. Everyone's future
plans have been completely put on hold or cancelled to a large degree.
We all miss the sense of normalcy and human sense of community from
being isolated for so long.

For this unique semester, we are going to prioritize supporting each
other as human beings during this crazy era, and use simple, accessible
solutions that make sense for the most people, and above all, to be
flexible. I have designed the course to maintain some common structure
but be flexible to your varied needs. Please see the
\protect\hyperlink{policies-and-expectations}{policies and expectations
below}. I hope you use this course as an opportunity to escape the
boredom and insanity of social isolation, and to help keep interest in
understanding the world around us.

If you tell me you're having trouble, I will do whatever I can to help,
and not judge you or think less of you. I hope you will extend me the
same courtesy.

\hypertarget{course-objectives}{%
\section*{Course objectives}\label{course-objectives}}
\addcontentsline{toc}{section}{Course objectives}

\textbf{By the end of this course,} you will:

\begin{itemize}
\tightlist
\item
  Predict the consequences of various laws, institutions, and customs
\item
  Predict what law will emerge under given conditions
\item
  Determine whether law \emph{is} economically efficient, and discuss
  whether the law \emph{should} be economically efficient
\item
  Derive the economic functions of key concepts, legal maxims, and rules
  in the substantive areas of law: torts, property, contracts, and
  criminal law
\item
  Identify the sources of law in the United States
\item
  Identify the key legal institutions of the United States
\end{itemize}

Given these objectives, this course fulfills all three of the
\href{https://www.hood.edu/academics/departments/george-b-delaplaine-jr-school-business/student-learning-outcomes}{learning
outcomes} for the Economics B.A. program in the George B. Delaplaine,
Jr.~School of Business:

\begin{itemize}
\tightlist
\item
  Use quantitative tools and techniques in the preparation,
  interpretation, analysis and presentation of data and information for
  problem solving and decision making {[}\ldots{]}
\item
  Apply economic reasoning and models to understand and analyze problems
  of public policy {[}\ldots{]}
\item
  Demonstrate effective oral and written communications skills for
  personal and professional success {[}\ldots{]}
\end{itemize}

\textbf{My standard disclaimer:} This class may challenge many of your
existing beliefs and conceptions about how the world works, and how it
should work. This is the most important and exciting part of a liberal
arts education. This does not mean that I want to make you to see
everything ``my way.'' In fact, if you come out of this class thinking
exactly like me, then I have probably failed you as a teacher. To the
best of my ability, I keep my opinions to myself unless relevant for
purposes of discussion, and I respect and invite you to reach your own
conclusions on all matters.

\textbf{Fair warning:} \textbf{Economics can be difficult.} Using the
economic way of thinking is a skill, it is literally retraining your
brain to interpret and analyze the world in a novel way, and is not
something that can be memorized. I will do my best to make this class
intuitive and helpful, if not interesting. If at any point you find
yourself struggling in this course for any reason, please come see me.
Do not suffer in silence! Coming to see me for help does not diminish my
view of you, in fact I will hold you in \emph{higher} regard for
understanding your own needs and taking charge of your own learning.
There are also a some fantastic resources on campus, such as the
\href{http://www.hood.edu/campus-services/academic-services/index.html}{Center
for Academic Achievement and Retention (CAAR)} and the
\href{http://www.hood.edu/library/}{Beneficial-Hodson Library}.

See my \href{http://lawS21.classes.ryansafner.com/reference\#tips}{tips
for success in this course}.

\hypertarget{required-course-materials}{%
\section*{Required Course materials}\label{required-course-materials}}
\addcontentsline{toc}{section}{Required Course materials}

This course requires regular online internet access. If you know you
will be unable to access the internet regularly, please let me know and
we can make arrangements.

You can find all course materials at my \textbf{dedicated website} for
this course:
\href{https://lawS21.classes.ryansafner.com}{lawS21.classes.ryansafner.com}.
Links to the website are posted on our Blackboard course page. Please
familiarize yourself with the website, see that it contains this
\href{https://lawS21.classes.ryansafner.com/syllabus/}{syllabus}, guides
for your
\href{https://lawS21.classes.ryansafner.com/reference/}{reference}, and
our \href{https://lawS21.classes.ryansafner.com/schedule/}{schedule}. On
the schedule page, you can find each module with its own class page
(\textbf{start there!}) along with all related readings, lecture slides,
practice problems, and assignments.

My lecture slides will be shared with you, and serve as your primary
resource, but our main ``textbook'' below is \textbf{recommended} as the
next best resource and will be available from the campus bookstore. I
will discuss more about textbooks and materials in the first module.

\hypertarget{books}{%
\subsection*{Books}\label{books}}
\addcontentsline{toc}{subsection}{Books}

\begin{enumerate}
\def\labelenumi{\arabic{enumi}.}
\item
  Cooter, Robert, and Thomas Ulen, 2012,
  \href{https://lawcat.berkeley.edu/record/1127400?ln=en}{\emph{Law \&
  Economics}}, 6\textsuperscript{th} ed., New York:
  Addison-Wesley\footnote{A free \& legal PDF version is available at
    the link.}
\item
  Friedman, David D, 2000, \emph{Law's Order: What Economics Has to do
  with Law and Why it Matters}, Princeton, NJ: Princeton University
  Press
\end{enumerate}

The first book is \textbf{required} but is out of print. Fortunately,
you can find old hard copies on Amazon, but ownership has reverted to
the authors, who have released their book as
\href{https://lawcat.berkeley.edu/record/1127400?ln=en}{\textbf{a free
PDF online}}. This will serve as your main textbook for the course, and
we will largely follow it in order.

The second is one of my favorite books on economics and law, which will
help you understand some of the concepts through creative puzzles. This
book I merely \textbf{recommend}, as we will not necessarily read from
it all the time, or in order.

I have no financial stake in requiring you to purchase either book. You
are welcome to use previous version of books, but carefully verify the
reading assignments, as the material may be different across versions.

\hypertarget{assignments-and-grades}{%
\section*{Assignments and Grades}\label{assignments-and-grades}}
\addcontentsline{toc}{section}{Assignments and Grades}

Your final course grade is the weighted average of the following
assignments. You can find general descriptions for all the assignments
on the
\href{http://lawS21.classes.ryansafner.com/assignments/}{assignments
page} and more specific information and examples on each assignment's
page on the
\href{http://lawS21.classes.ryansafner.com/schedule/}{schedule page}.

\begin{center}

\begin{tabular}{lll}
\toprule
Frequency & Assignment & Weight\\
\midrule
1 & Midterm & 25\%\\
1 & Final & 25\%\\
n & Homework (Average) & 20\%\\
1 & Term Paper & 20\%\\
n & Participation/Discussion Boards & 10\%\\
\bottomrule
\end{tabular}
\end{center}

Each assignment is graded on a 100 point scale. Letter-grade equivalents
are based on the following scale:

\begin{center}

\begin{tabular}{llll}
\toprule
Grade & Range & Grade1 & Range1\\
\midrule
A & 93–100\% & C & 73–76\%\\
A− & 90–92\% & C− & 70–72\%\\
B+ & 87–89\% & D+ & 67–69\%\\
B & 83–86\% & D & 63–66\%\\
B− & 80–82\% & D− & 60–62\%\\
\addlinespace
C+ & 77–79\% & F & < 60\%\\
\bottomrule
\end{tabular}
\end{center}

See also my
\href{https://ryansafner.shinyapps.io/law_grade_calculator/}{
\texttt{Grade\ Calculator}} app where you can calculate your overall
grade using existing assignment grades and forecast ``what if''
scenarios.

These grades are firm cutoffs, but I do of course round upwards
\((\geq\) 0.5) for final grades. A necessary reminder, as an academic, I
am not in the business of \emph{giving} out grades, I merely report the
grade that you \emph{earn}. I will not alter your grade unless you
provide a reasonable argument that I am in error (which does happen from
time to time).

\hypertarget{homeworks}{%
\subsection*{Homeworks}\label{homeworks}}
\addcontentsline{toc}{subsection}{Homeworks}

There will be several homework assignments over the semester. These
questions will help check your understanding and mastery of the
material, and will be a combination of quantitative, graph, and short
answer questions of examples. You may collaborate with other students to
work on homeworks, but each person must turn in an individual
assignment. I will grade homeworks follows: 70\% of the grade for
completion, and 30\% for one randomly selected question. This is to
reward students for putting in a full faith effort to try to reach an
answer, even if not every answer ends up being correct.

Homeworks are generally due by 11:59 PM EST Sunday by submission to
Blackboard Assignments.

Please \textbf{type} your answers to the following questions in a
document and \textbf{save as a PDF}\footnote{In MS Word, or Pages, or
  most word processing software, File -\textgreater{} Save As
  -\textgreater{} PDF, or File -\textgreater{} Export -\textgreater{}
  PDF.} to \textbf{upload on Blackboard} under Assignments. You may
handwrite answers if you will be able to scan/photograph \& convert them
\textbf{to a single PDF}, if they are easily readable, but this is
\emph{not preferred}. See my
\href{https://microS21.classes.ryansafner.com/resources/\#how-to-make-a-pdf-for-submitting-assignments}{guide
to making a PDF} - an essential skill in the modern world. If you are
handwriting answers, you may print the \texttt{pdf} above and write on
it, or just write on a piece of paper (we only need your answers).

For the few questions that ask you to draw a \textbf{graph}, \emph{try}
to do so \emph{on your computer} (use MS Paint, the drawing tools in MS
Word or MS Powerpoint, plot points in MS Excel, drawing/notetaking apps,
etc.), and save it as an image to include on your homework document.
Again, they need not be perfect or to scale, just show that you
understand the broad idea. Being able to understand and sketch the
graphs is still a very important and useful skill! If all else fails, I
will be lenient in grading graph questions if you are unable to
technologically include a graph.

I will grade homeworks 70\% for completion, and for the remaining 30\%,
and one question will be graded for accuracy - so it is best that you
try every problem, even if you are unsure how to complete it accurately.

\hypertarget{exams}{%
\subsection*{Exams}\label{exams}}
\addcontentsline{toc}{subsection}{Exams}

The midterm exam will be released \textbf{on Blackboard} as a timed
assignment. You will have 2 hours once you open the exam on Blackboard.
You will not need the whole time, I have given you some extra to
accommodate the difficulties of taking an exam at home. Please pick a
time to take it where you know you will have 2 hours. You may close the
exam page and come back to it, but the timer will continue to run once
the exam is first opened.

The final exam will be a take-home series of essay questions.

\hypertarget{term-paper}{%
\subsection*{Term Paper}\label{term-paper}}
\addcontentsline{toc}{subsection}{Term Paper}

I am looking for you to write a paper that analyzes a particular
law/regulation, institution, custom/practice, or legal/policy issue
using the economic analysis of law we develop in this course. For
example --- why does it exist? what are the consequences, and how does
it affect people's incentives? is it efficient? etc. For more details,
view the \href{assignment/paper}{assignment's page}.

\hypertarget{participationdiscussion-board}{%
\subsection*{Participation/Discussion
Board}\label{participationdiscussion-board}}
\addcontentsline{toc}{subsection}{Participation/Discussion Board}

We will have a discussion board thread on Blackboard open most weeks in
the course. I will let you know at the beginning of the week if there
is, or is not, a required board open. You will be expected to contribute
to the discussion board at least twice in the week. Your weekly
contribution will be graded out of 5 points. At the end of the semester,
I will apply the \emph{average} of your weekly participation grades to
apply (10\%) towards your final course grade.

I am interested in your thoughts, reactions, comments, and questions
about any of the material (lectures and/or readings). You do not need to
write more than a paragraph. Anything more than that, including
continuing to reply to each others' thoughts, questions, or comments,
(which I strongly hope you do!) is solely based on your own interest and
curiosity. I will jump in to answer questions the group is stuck on,
give my two cents, and stir the pot as needed. I strongly hope we still
keep a conversation going and can learn from each other, that was always
my goal, not to lecture at you! If you crave visual human contact, you
can submit your comments/reactions in the form of a short video, and we
can try that out!

See the \href{/assignments/}{assignment page} for my grading rubric for
each week (when required).

Notice it is possible to get above 5 points for a truly remarkable week
of contributions, but I give these sparingly.

\hypertarget{no-extra-credit-is-available}{%
\subsection*{No extra credit is
available}\label{no-extra-credit-is-available}}
\addcontentsline{toc}{subsection}{No extra credit is available}

\hypertarget{policies-and-expectations}{%
\section*{Policies and Expectations}\label{policies-and-expectations}}
\addcontentsline{toc}{section}{Policies and Expectations}

This syllabus is a contract between you, the student, and me, your
instructor. It has been carefully and deliberately thought
out\footnote{A syllabus can and will be used as a legal document for
  disputes tried at a court of law. Ask me how I know.}, and I will
uphold my end of the agreement and expect you to uphold yours.

In the language of game theory, this syllabus is my commitment device. I
am a very understanding person, and I know that exceptions to rules
often need to be made for students. However, to be \emph{fair} to
\emph{all} students the syllabus artificially constrains my ability to
make exceptions at a whim for anyone. This prevents clever students from
exploiting my congenial personality at everyone else's expense. Please
read and familiarize yourself with the course policies and expectations
of you. Chances are, if you have a question, it is answered herein.

\hypertarget{online-attendance-and-participation}{%
\subsection*{Online Attendance and
Participation}\label{online-attendance-and-participation}}
\addcontentsline{toc}{subsection}{Online Attendance and Participation}

This is a hybrid course with synchronous (live) and asynchronous (on
your own time) parts.

You are generally expected to join (online via Zoom) our
\textbf{synchronous} class sessions unless circumstances prevent you
from doing so. Day-to-day attendance is not graded per se, but I
strongly recommend you join in all live sessions in which you are able,
since we all can provide live feedback and I can answer questions and
address concerns as soon as they come up. You will also benefit from a
rigid schedule and shared community.

If you are unable to make a particular class, you generally do not need
to let me know. \textbf{The videos from all class sessions are posted on
Blackboard} so please review videos of classes you were unable to attend
live.

All assignmnents are able to be completed \textbf{asynchronously} during
the week, and are \textbf{generally due by 11:59PM Sunday each week} to
allow you flexibility in your hectic schedules.

\hypertarget{late-assignments}{%
\subsection*{Late Assignments}\label{late-assignments}}
\addcontentsline{toc}{subsection}{Late Assignments}

I will accept late assignments, but will subtract a specified amount of
points as a penalty. Even if it is the last week of the semester, I
encourage you to turn in late work: some points are better than no
points!

\textbf{Homeworks}: If you turn in a homework after it is due but before
it is graded or the answer key posted, I generally will not take off any
points. However, \textbf{if you turn in a homework \emph{after} the
answer key is posted, I will automatically deduct 20 points (so the
maximum grade you can earn on it is an 80).}

\textbf{Exams}: If you know that you will be unable to complete an
\emph{exam} as scheduled for a legitimate reason, please notify me at
least \emph{one week} in advance, and we will schedule a make-up exam
date. Failure to do so, including desperate attempts to make
arrangements only \emph{after} the exam will result in a grade of 0 and
little sympathy. I reserve the right to re-weight other assignments for
students who I believe are legitimately unable to complete a particular
assignment.

\hypertarget{grading}{%
\subsection*{Grading}\label{grading}}
\addcontentsline{toc}{subsection}{Grading}

I will try my best to post grades on Blackboard's Grading Center and
return graded assignments to you within about one week of you turning
them in. There will be exceptions. Where applicable, I will post answer
keys once I know most homeworks are turned in (see Late Assignments
above for penalties). Blackboard's Grading Center is the place to look
for your most up-to-date grades. See also my
\href{https://ryansafner.shinyapps.io/law_grade_calculator/}{
\texttt{Grade\ Calculator}} app where you can calculate your overall
grade using existing assignment grades and forecast ``what if''
scenarios.

\hypertarget{communication-email-slack-and-virtual-office-hours}{%
\subsubsection*{Communication: Email, Slack, and Virtual Office
Hours}\label{communication-email-slack-and-virtual-office-hours}}
\addcontentsline{toc}{subsubsection}{Communication: Email, Slack, and
Virtual Office Hours}

Students must regularly monitor their \textbf{Hood email accounts} to
receive important college information, including messages related to
this class. Email through the Blackboard system is my main method of
communicating announcements and deadlines regarding your assignments.
\textbf{Please do not reply to any automated Blackboard emails - I may
not recieve it!}. My Hood email (\texttt{safner@hood.edu}) is the best
means of contacting me. I will do my best to respond within 24 hours. If
I do not reply within 48 hours, do not take it personally, and
\emph{feel free to send a follow up email} in the very likely event that
I genuinely did not see your original message.

Our \href{https://hoodcollegeeconomics.slack.com}{slack channel} is
available to all students and faculty in Economics and Business. I have
invited all of my classes and advisees. It will not be extended to
non-Business/Economics students or faculty. All users must use their
\textbf{hood emails} and \textbf{true first and last names}. Each course
has its own channel, exclusive for verified students in the course, and
myself, by my invite only. As a third party platform, you agree to its
Terms of Service. I have created this space as a way to stay connected,
to help one another, and to foster community. Behaviors such as posting
inappropriate content, harassing others, or engaging in academic
dishonesty, to be determined solely at my discretion, will result in one
warning, the content will be deleted, and subsequent behavior will
result in a ban.

I will host general \textbf{``office hours''} on Zoom. You can join in
with video, audio, and/or chat, whichever you feel comfortable with. Of
course, if you are not available during those times, we can schedule our
own time if you prefer this method over email or Slack. If you want to
go over material from class, please have \emph{specific} questions you
want help with. I am not in the business of giving private lectures
(particularly if you missed class without a valid excuse).

Watch the excellent and accurate video
\href{https://vimeo.com/270014784}{explaining office hours} (on website
syllabus page).

\hypertarget{netiquette}{%
\subsection*{Netiquette}\label{netiquette}}
\addcontentsline{toc}{subsection}{Netiquette}

When using Zoom and Slack, please follow appropriate internet etiquette
(``Netiquette''). Written communications, like blog posts or use of the
Zoom chat, lacks important nonverbal cues (such as body language, tone
of voice, sarcasm, etc).

Above all else, please respect one another and think/reread carefully
about how others may see your post before you submit a comment. You are
expected to disagree and have different opinions, this is inherently
valuable in a discussion. Please be civil and constructive in responding
to others' comments: writing \emph{``have you considered `X'?''} is a
lot more helpful to all involved than just writing \emph{``well you're
just wrong.''}

Posting content that is wilfully incindiary, illegal, or that
constitutes academic dishonesty (such as plagarism) will automatically
earn a grade of 0 and may be elevated to other authorities on campus.

When using the chat function on Zoom or public Slack channels, please
treat it as official course communications, even though I may not be
grading it. It may be a quick and informal tool - don't feel you need to
worry about spelling or perfect grammar - but please try to avoid
\emph{too} informal ``text-speak'' (i.e.~say ``That's good for you''
instead of ``thas good 4 u'').

\hypertarget{privacy}{%
\subsection*{Privacy}\label{privacy}}
\addcontentsline{toc}{subsection}{Privacy}

\href{https://www.execvision.io/blog/maryland-call-recording-laws/}{Maryland
law}
\href{https://law.justia.com/codes/maryland/2005/gcj/10-402.html}{requires}
all parties consent for a conversation or meeting to be recorded. If you
join in, and certainly if you participate, \textbf{you are consenting to
be recorded.} However, as described below, videos are \emph{not
accessible} beyond our class.

Live lectures are recorded on Zoom and posted to Blackboard via Panopto,
a secure course management system for video. Among other nice features
(such as multiple video screens, close captioning, and time-stamped
search functions!), Panopto is authenticated via your Blackboard
credentials, ensuring that \emph{our course videos are not accessible to
the open internet.}

For the privacy of your peers, and to foster an environment of trust and
academic freedom to explore ideas, \textbf{do not record our course
lectures or discussions.} You are already getting my official copies.

The
\href{https://www2.ed.gov/policy/gen/guid/fpco/ferpa/index.html}{Family
Educational Rights and Privacy Act} prevents me from disclosing or
discussing any student information, including grades and records about
student performance. If the student is at least 18 years of age,
\emph{parents (or spouses) do not have a right to obtain this
information}, except with consent by the student.

Many of you may be tuning in remotely, living with parents, and may have
occasional interruptions due to sharing a space. This is normal and
fine, but know that I will protect your privacy and not discuss your
performance when parents (or anyone other than you, for that matter) are
present, without your explicit consent.

\hypertarget{enrollment}{%
\subsection*{Enrollment}\label{enrollment}}
\addcontentsline{toc}{subsection}{Enrollment}

Students are responsible for verifying their enrollment in this class.
The last day to add or drop this class with no penalty is
\textbf{Wednesday, February 10}. Be aware of
\href{https://www.hood.edu/offices-services/registrars-office/academic-calendar}{important
dates}.

\hypertarget{honor-code}{%
\subsection*{Honor Code}\label{honor-code}}
\addcontentsline{toc}{subsection}{Honor Code}

Hood College has an Academic Honor Code which requires all members of
this community to maintain the highest standards of academic honesty and
integrity. Cheating, plagiarism, lying, and stealing are all prohibited.
All violations of the Honor Code are taken seriously, will be reported
to appropriate authority, and may result in severe penalties, including
expulsion from the college. See
\href{http://hood.smartcatalogiq.com/en/2016-2017/Catalog/The-Spirit-of-Hood/The-Academic-Honor-Code-and-Code-of-Conduct}{here}
for more detailed information.

\hypertarget{van-halen-and-mms}{%
\subsection*{Van Halen and M\&Ms}\label{van-halen-and-mms}}
\addcontentsline{toc}{subsection}{Van Halen and M\&Ms}

When you have completed reading the syllabus, email me a picture of the
band Van Halen and a picture of a bowl of M\&Ms.~If you do this
\emph{before} the date of the first exam, you will get bonus points on
the exam.\footnote{If 75-100\% of the class does this, you each get 2
  points. If 50-75\% of the class does this, you each get 4 points. If
  25-50\% of the class does this, you each get 6 points. If 0-25\% of
  the class does this, you each get 8 points.} Yes, this is real.

\hypertarget{accessibility-equity-and-accommodations}{%
\subsection*{Accessibility, Equity, and
Accommodations}\label{accessibility-equity-and-accommodations}}
\addcontentsline{toc}{subsection}{Accessibility, Equity, and
Accommodations}

College courses can, and should, be challenging and bring you out of
your comfort zone in a safe and equitable environment. If, however, you
feel at any point in the semester that certain assignments or aspects of
the course will be disproportionately uncomfortable or burdensome for
you due to any factor beyond your control, please come see me or email
me. I am a very understanding person and am happy to work out a solution
together. I reserve the right to modify and reweight assignments at my
sole discretion for students that I belive would legitimately be at a
disadvantage, through no fault of their own, to complete them as
described.

If you are unable to afford required textbooks or other resources for
any reason, come see me and we can find a solution that works for you.

This course is intended to be accessible for all students, including
those with mental, physical, or cognitive disabilities, illness,
injuries, impairments, or any other condition that tends to negatively
affect one's equal access to education. If at any point in the term, you
find yourself not able to fully access the space, content, and
experience of this course, you are welcome to contact me to discuss your
specific needs. I also encourage you to contact the
\href{https://www.hood.edu/academics/josephine-steiner-center-academic-achievement-retention/accessibility-services}{Office
of Accessibility Services} (301-696-3421). If you have a diagnosis or
history of accommodations in high school or previous postsecondary
institutions, Accessibility Services can help you document your needs
and create an accommodation plan. By making a plan through Accessibility
Services, you can ensure appropriate accommodations without disclosing
your condition or diagnosis to course instructors.

\hypertarget{schedule}{%
\section*{Schedule}\label{schedule}}
\addcontentsline{toc}{section}{Schedule}

\textbf{You can find a full schedule} with much more details, including
the readings, appendices, and other further resources for each class
meeting on the
\href{http://lawS21.classes.ryansafner.com/schedule/}{course website's
schedule page}.

\end{document}